%objetivos.tex

\section{Diary of work}
\label{sec:objetivos}

\subsection{First week}
Fist week, my partners explained me all extension of the proyect, what and how we will do it.
I read a lot of documentation of the proyect Geo-Cloud.
I took a look for the proyect planification and I think if there were some to change or troubles, but it has been well-done.

\subsection{Second week}
I registered in the BonFIRE's official page\cite{BF} and the iMinds' main page\cite{VW}. The last page gave me an certificate in format pem for to get access the iMinds TestBed. In both pages, I had to push my public RSA key for getting access in its testbeds.
I had read with BonFIRE's documentation. I tried to start an experiment with some storage and compute resources successfully.
Also, I began with Orchestrator's designing. I made the basic architecture and drew the class diagrams.
Last, I built the Orchestrator and  began to made some test but are incomplete. In the first phase of Orchestrator, we have a functional Orchestrator but not definetly so we will have to re-build and add more complexity and funcionalities.

\subsection{Third week}

First, I read a lot about Geo-Server\cite{GS} and Geo-Network\cite{GN}. We understood how that platforms work so we decided to work with Geo-Server and a Csw plugin for it. I installed Apache Tomcat and via War files, I installed Geo-Server in Tomcat.
When Geo-Server was running, I started with designing of the communication between Geo-Server and Orchestrator. This communication is easy so only the Orchestrator sends the result of the processing to Geo-Server.
I tried to generate an experiment with jFed\cite{jFed}. I had to install java manually because the system didn't recognise the java binary, so I downloaded it of the official page and installed it manually. In addition, I installed icedtea-plugin for firefox in order to it can open java binaries.

\subsection{Fourth week}
I installed Nepi\cite{Nepi} and python v7 because with version 6, Nepi has lots of errors.
I made an experiment template in order to develop the VirtualWall slice. This template will serve to us for making all experiments for all escenaries.
I tried lots of times to execute an PlanetLab\cite{PLE} script getting some errors. So, I had a meeting with Alina and Lucía for solve this exceptions.
The meeting is in a document in wiki.
Also, I tried with a VirtualWall script too but I got the same result.
The next week, we have to develop the PlanetLab scrip with a test execution of the scipt and verify that works fine.
In addition, we have to develop the VirtualWall scripts (I think they are so easiest) and testing them.

\subsection{Fifth week}
\subsubsection{Fist day}

I could test a planet lab node in jFed and deploying a ping application.
I looked for a method than evaluates the net features.
I realised the first layer of Planet Lab testbed in jFed but the most of host I can't select into jFed or they are off.
Design the database for satellite data.
Installation of mysql-workbench, python-mysqldb.
I done the database script that inicialize the ground stations, the cases but I haven't done the satellites yet.


\subsubsection{Rest of week}
I could deploy an experiment with jFed and Nepi.
The next days, I tried to create in BonFire platform, an experiment with some resources and I created the network topologies in layer 1. The problems were that I deploy a network resource in iMinds, but I not could attach a storage resource or an compute resource too.
We modified and completed the Orchestrator and Archive report.
Also, I have the data base script for completion and the satellite script were started.

\subsection{Sixth week}

This week I have already finished the script initialization of all data in data base.
Also, I remastered and improved the Orchestrator report adding new images and explaining a bit more the use cases and the detail funtionalities.

Finally, on thursday, I started to design and to develop the satellite software. This software is complicated because it is necessary to simulate each zone ( as if this zone is interesting as no it is). The interesting areas and the visibility cones are simulated using the scheduler module of python.
Like a real scheduler, I took the time of each zone and, calculating the relative time respect of beginning execution time, and done the schedule with all areas, which own priority that could be high o low.

\subsection{6-april-14}
This last weeks I have tried to create and execute de PLE experiment. I had some troubles with the PLE nodes because some are unreachables, or the ssh connection were refused, so those nodes 
must be changed for others that are full functional.
The nodes field register must be updated with the finally selected nodes.
The execution of the experiment was succesfully and I could obtain some information as loss rate and delay. The execution takes 12 hours sending iformation to senver.
The iperf software was used for to do this, and the client command has to be explained in the documentation.
The next step is to do the PLE experiment documentation to explain:
-- the iperf software
-- why the parameters in the client and server has been choosen
-- Nepi
-- Script using Nepi
-- time of execution
-- obtained results
-- graphical results : average bandwith per node, average loss rate per node and average delay per node
----- bandwith behaviour per node for a 12 hours
----- " loss rate and jitter

Also, we wrote two abstracts for the FOOS congres.

