%objetivos.tex

\section{Diary of work}
\label{sec:objetivos}

\subsection{First week}
Fist week, my partners explained me all extension of the proyect, what and how we will do it.
I read a lot of documentation of the proyect Geo-Cloud.
I took a look for the proyect planification and I think if there were some to change or troubles, but it has been well-done.

\subsection{Second week}
I registered in the BonFIRE's official page\cite{BF} and the iMinds' main page\cite{VW}. The last page gave me an certificate in format pem for to get access the iMinds TestBed. In both pages, I had to push my public RSA key for getting access in its testbeds.
I had read with BonFIRE's documentation. I tried to start an experiment with some storage and compute resources successfully.
Also, I began with Orchestrator's designing. I made the basic architecture and drew the class diagrams.
Last, I built the Orchestrator and  began to made some test but are incomplete. In the first phase of Orchestrator, we have a functional Orchestrator but not definetly so we will have to re-build and add more complexity and funcionalities.

\subsection{Third week}

First, I read a lot about Geo-Server\cite{GS} and Geo-Network\cite{GN}. We understood how that platforms work so we decided to work with Geo-Server and a Csw plugin for it. I installed Apache Tomcat and via War files, I installed Geo-Server in Tomcat.
When Geo-Server was running, I started with designing of the communication between Geo-Server and Orchestrator. This communication is easy so only the Orchestrator sends the result of the processing to Geo-Server.
I tried to generate an experiment with jFed\cite{jFed}. I had to install java manually because the system didn't recognise the java binary, so I downloaded it of the official page and installed it manually. In addition, I installed icedtea-plugin for firefox in order to it can open java binaries.

\subsection{Fourth week}
I installed Nepi\cite{Nepi} and python v7 because with version 6, Nepi has lots of errors.
I made an experiment template in order to develop the VirtualWall slice. This template will serve to us for making all experiments for all escenaries.
I tried lots of times to execute an PlanetLab\cite{PLE} script getting some errors. So, I had a meeting with Alina and Lucía for solve this exceptions.
The meeting is in a document in wiki.
Also, I tried with a VirtualWall script too but I got the same result.
The next week, we have to develop the PlanetLab scrip with a test execution of the scipt and verify that works fine.
In addition, we have to develop the VirtualWall scripts (I think they are so easiest) and testing them.

\subsection{Fifth week}
\subsubsection{Fist day}

I could test a planet lab node in jFed and deploying a ping application.
I looked for a method than evaluates the net features.
I realised the first layer of Planet Lab testbed in jFed but the most of host I can't select into jFed or they are off.
Design the database for satellite data.
Installation of mysql-workbench, python-mysqldb.
I done the database script that inicialize the ground stations, the cases but I haven't done the satellites yet.


\subsubsection{Rest of week}
I could deploy an experiment with jFed and Nepi.
The next days, I tried to create in BonFire platform, an experiment with some resources and I created the network topologies in layer 1. The problems were that I deploy a network resource in iMinds, but I not could attach a storage resource or an compute resource too.
We modified and completed the Orchestrator and Archive report.
Also, I have the data base script for completion and the satellite script were started.
