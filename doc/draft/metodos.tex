%Métodos y fases de trabajo

\section{Métodos y fases de trabajo}
\label{sec:metodos}

Para llevar a cabo los objetivos propuestos en este documento se dividirá el proyecto en las siguientes fases:

\begin{enumerate}
\item \textbf{Estudio del estado del arte:} En esta fase será necesario conocer los distintos testbeds proporcionados por el consorcio \emph{Fed4FIRE} y comprender bien  los aspectos relativos a su funcionamiento.
Se realizará un estudio para obtener las métricas a tener en cuenta a la hora de la implementación de la simulación.
Se realizará un estudio en profundidad sobre las posibilidades que ofrece BonFIRE para la creación y control de los experimentos.
Se estudiará la plataforma Virtual Wall para la simulación de nodos.
Se estudiará la plataforma PlanetLab Europe para la obtención de las métricas reales y poder ser cercano a la realidad.
Se estudiará el centro de proceso de datos de la empresa Elecnor Deimos y más concretamente el núcleo de procesamiento de los datos raw para poder implementarlo en la nube que nos proporciona BonFIRE.



\item\textbf{Estudio de la viabilidad:}
El proyecto se adaptará realizando pruebas para verificar el correcto funcionamiento del mismo y su
puesta en marcha. Se dispondrán de los distintos testbeds y el computador de desarrollo donde poder probar la plataforma. Para el uso óptimo de la aplicación
es muy recomendable determinar qué requisitos, tanto hardware como software, serán necesarios.

\item\textbf{Análisis de la arquitectura del sistema:}
Partiendo de los requisitos más importantes que debe cumplir este proyecto, se realizará un análisis más concreto de los componentes
que formarán la arquitectura teniendo en cuenta los principios de modularización y extensibilidad.

\item\textbf{Diseño de la arquitectura del sistema:}
En esta fase se intentará concretar, a un nivel más cercano a la implementación, cada una de las clases que conformarán la arquitectura del sistema.
Se podrá utilizar un lenguaje específico para este propósito como \emph{UML}. Esta fase se realizará de forma concurrente junto a las fases de desarrollo
y pruebas, realimentándose para mejorar el diseño de la arquitectura.


\item\textbf{Desarrollo del sistema:}
Se dará comienzo a la implementación de todos los módulos que formarán el sistema, en el lenguaje y formato escogido, siguiendo la arquitectura del sistema.
Se usará una metodología iterativa incremental para el desarrollo del mismo.

\item \textbf{Pruebas del sistema:}
Se tratará de ir realizando pruebas tanto de caja negra como de caja blanca de cada una de las clases implementadas.
Cuando se termine el sistema, se harán pruebas de integración de todos los componentes y su funcionamiento.

\item\textbf{Evaluación de los resultados:}

Se evaluará la ganancia en eficiencia respecto a un sistema de procesamiento de imágenes raw en un CPD el cual no se beneficie de los beneficios de la ejecución y distribución en la nube.


\item\textbf{Documentación del proyecto:}
De forma paralela al desarrollo del proyecto se irá generando la documentación del mismo (diagramas de clases, de análisis, requisitos, manual de usuario, étc.) para completarla al final con las conclusiones y las
pruebas finales que se realicen.


\end{enumerate}
% \nocite{*}







% Local Variables:
%   coding: utf-8
%   fill-column: 90
%   mode: flyspell
%   ispell-local-dictionary: "castellano"
%   mode: latex
%   TeX-master: "main"
% End:
