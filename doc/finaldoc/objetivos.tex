\chapter{Objectives}
\label{chap:objetivos}
%\drop{I}{n} this chapter the  objectives of this project are described.
\drop{T}{he} main objective of the GEO-Cloud project is the modelling and
implementation of a close to real world Earth Observation System in
\emph{Fed4FIRE} cloud. 

\section{Specific Objectives}

The main aim of this project is defined regarding a series of functional
objectives as follows:
\begin{itemize}

\item {\textbf{To implement a data centre in the multicloud BonFIRE testbed}: This
  infrastructure for \ac{EO} provides the ingestion of raw data, on demand processing the raw
  data to obtain images, and later storing and cataloguing transparently,
  dynamically and automatically. For this purpose this components are needed:
}
\begin{itemize}

\item{\textbf{Cloud Orchestrator}: To manage the
  processing of images obtained from satellites and to control all the stages
  (ingestion of images, processing, storing and cataloguing).}
\item {\textbf{Archive and Catalogue Subsystem}: To archive, to catalogue and to
    provide to end-users, 
 the results obtained from the processing stage.}
\item{\textbf{Processors Module}: To process the raw data sent from
  the Orchestrator component obtaining processed images from the Earth. The product
  processors are provided and owned by \emph{Elecnor Deimos} company, so their implementation are treated as a black box.}
\end{itemize}

\item \textbf{To implement a Space Simulator in Virtual Wall testbed:} This
  software simulates: the satellite
  constellation getting images and downloading them into the ground stations; and ground
  stations receiving the raw data sent by the satellites and sending it to cloud
  infrastructure. In addition, the connections
  between the satellite constellation and ground stations and between the ground
  stations and cloud infrastructure are also simulated. 

\item {\textbf{To obtain the network impairments values using PlanetLab testbed}: In
    order to implement a closest to reality experiment, the real values of
    loss-rate, bandwidth and \ac{RTT} of both the satellites-ground stations and
    ground stations-cloud infrastructure networks, are obtained. Then, the
    impairments of the networks implemented using the \vw testbed, are updated.}

\item {\textbf{To create the experiment}: By combining and integrating
  the developed in the three testbeds mentioned before.}

\item {\textbf{To create a Graphical User Interface}: This \ac{GUI} provides
  the deploy,execution and stop, easily and automatically.}

\item {\textbf{To validate the experiment}: Simulating of a scenario to validate the
  entire system obtaining, downloading, processing, storing and cataloguing
  \ac{EO} images. In addition,  to evaluate if future internet cloud computing
  and networks provide viable solutions for conventional \ac{EO}
  Systems to establish the basis for the implementation of \ac{EO}
  infrastructures in
  cloud. }

\end{itemize}
