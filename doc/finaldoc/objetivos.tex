\chapter{Objectives}
\label{chap:objetivos}
In this chapter the objectives carryied out in this project are included.

\section{General Aim}

The GEO-Cloud project main objetive can be defined as the modeling and
implementation of a (antes -> experiment) close to real world \ac{EO} System in \emph{Fed4FIRE} to validate the \emph{Fed4FIRE} infrastructure.

\section{Specific Objectives}

The main objective of this project is defined regarding a serie of functional
requirements as follows:
\begin{itemize}
  
\item {\textbf{A Cloud Infrastructure in BonFIRE Testbed}: This
  infrastructure for \ac{EO} provides the ingestion of raw data, on demand processing the raw
  data to obtain images, and later storing and cataloging transparently,
  dinamically and automatically. For this purpouse this components are nedeed:
}
\begin{itemize}

\item{\textbf{Cloud Orchestrator}: It manages the
  processing of images obtained from satellites and controls all the stages
  (ingestion of images, processing, storing and cataloging). The images
  processors are provided by \emph{Elecnor Deimos} company, so their implementation is propietary and they are treated as a black box.}
\item {\textbf{Archive and Catalogue Subsystem}: The
  results obtained from the processing have to be available in the cloud
  infrastructure for end users.}
\item{\textbf{Processors Module}: It process the raw data sended from
  the Orchestrator component obtaining processed images from the Earth.}  
\end{itemize}
\item {\textbf{A profiling tool in PlanetLab}: for obtaining the real measures of
  parameters and metrics involved in the communications of all components as
  clients, ground stations, cloud and satellite constellation.}


\item {\textbf{Software in Virtuall Wall}: This sofware simulates the satellite
  constellation getting images and downloading into the ground stations, ground
  stations receiving the raw data sended by the satellites and end users
  accessing the resources in cloud through web services. Also, the connections
  between the ground stations and the cloud infrastructure have a set of
  impairments. This values of these impairments are obtained with the profiling tool in
  \pl in order to build a  realistic experiment.}

\item {\textbf{Creation of the experiment}: By combining and integrating
  the developed in the three testbeds mentioned before.}

\item {\textbf{Creation of a Graphical User Interface}: This \ac{GUI} provides
  the deploy,execution and stop automatically.} 

\item {\textbf{System Validation}: Simulation of a scenario to validate the
  entire system obtaining, downloading, processing, storing and cataloging
  \ac{EO} images. }

\item \textbf{To validate if future internet cloud computing
  and networks provide viable solutions for conventional Earth Observation
  Systems to establish the basis for the implmentation of EO infrastructrures in
  cloud.}

\end{itemize}
