\chapter{Source code}

Due to the extension of the source code (5.000 lines), it is solely included
in the CR-ROM attached to this document.

The distribution of the source in the implementation of the GeoCloud experiment
The implementation of the GEO-Cloud experiment consists of some parts:
\begin{itemize}
\item To create a Space System Simulator 
\item To implement a Virtual Wall experiment and to use jFed to deploy it.
\item To create in BonFIRE the Orchestrator, A\&C , Database and Processing Chain. 
\item To create setup scripts to initialize the nodes into BonFIRE and Virtual Wall.
\end{itemize}

In order to explain the following sections, the root directory of all files it
is namely ``GeoCloud''. Inside it there are two folders: ``doc'' which contains
the documentation and ``source'' which contain all source code developed in this
project.

\section{Space System Simulator files}

The Space System Simulator consists of a Satellite System Simulator and a Ground
Station System Simulator. The Satellite System Simulator is constituted by 17
Satellite Simulators executing at time. Also the Ground Station System Simulator
is constituted by 12 Ground Station Simulators executing at time. The source of
the software is located in \emph{source/vw/simulator} as follows:
\begin{itemize}
\item \emph{satellite.py}: source that Satellite Simulator.
\item \emph{groundstation.py}: source that Ground Station Simulator.
\item \emph{runGS.sh}: Bash script that deploys and starts the Ground Station Simulators.
\item \emph {runSat.sh}: Bash script that deploys and starts the Satellite Simulators.
\item \emph {clean.sh}: Bash script that cleans all log files and temporally files.
\end{itemize}

\section{Virtual Wall deployment files}

The files used to deploy the Spatial System Simulator in Virtual Wall are
located in \emph{source/vw} and they are the following:
\begin{itemize}
\item \emph{Description.rspec}: Specification in RSPEC format in order to jFed creates the experiment.  This file contains the nodes reservation and some commands to execute after the node initialization.
\item \emph{install\_ftp.sh}: Installation of the ftp server on the node.
\item \emph{pus\_ip.sh}: Script that puts the IP database in a file on the node.
\end{itemize}

\section{Orchestrator, Archive and Catalogue, Database and Processing Chain
  files}

The BonFIRE is composed by the Orchestrator, A\&C, database and Processing Chain
modules. The Orchestrator, A\&C and Processing Chain software in the GEO-Cloud
project are in the Cloud. The Database is in the \bonfire platform but is only
used for hosting the hostnames which the ground station nodes have.

Such there are two implementations for the Cloud architecture, each one of them
it is located in a different location each other.

The first implementation it is located in \emph{source/bonfire} as follows:

\begin{itemize}
\item \emph{Orchestrator}: All source is located in \emph{source/bonfire/orchestrator} as follows:
\begin{itemize}
\item \emph{listener.py}: listener class used by Orchestrator component.
\item \emph{main.py}: contains the main method used by Orchestrator component.
\item \emph{orchestrator.py}: contains the class Orchestrator used by Orchestrator component.
\item \emph{processingChain.py}: contains the processingChain component used by Orchestrator.
\item \emph{Orchestrator.conf.xml}: contains the initial configuration for the Orchestrator in \ac{XML} format.
\item \emph{Load.py (Not used)}: Python class that obtains the CPU workload
  remotely.  
\end{itemize}
\item \emph{A\&C}: All source is located in \emph{source/bonfire/geoserver} as
  follows:
\begin{itemize}
\item \emph{install.sh}: bash script that install the A\&C. Must be executed in a BonFIRE node.
\item \emph{catalog\_pp.py}: this script is called by the processing chain in order to store and catalog the processed image into the A\&C module.
\item \emph{Install\_ftp.sh}: bash script that install a ftp server on the node.
\end{itemize}
\item \emph{Product Processors}: Source is located in \emph{source/bonfire/processingChain} as follows:
\begin{itemize}
\item \emph{PPscript.sh}: bash script that it is called by the Orchestrator in
  order to process an image. 
\end{itemize}
\end{itemize}

The second implementation using \emph{ZeroC ICE} is located in
\emph{source/ice/} as follows:

\begin{itemize}

\item \emph{app}: this folder contains the distributed application for its
  deployment.
\item \emph{cfg}: it contains the configuration files for deploying the nodes
  using \emph{IceGrid} service.
\item \emph{certs}: it contains the certificates for authenticating connections.
\item \emph{bin}: it contains the clean, start and stop scripts.
\item \emph{test}: this folder contains the developed test in order to check the
  funcionality of the components.
\item \emph{src}: this directory contains the source files for each component of
  the cloud.
\end{itemize}

The Database files are located in \emph{source/database/modelDatabase} and
\emph{source/database/scenaries}.
\begin{itemize}
\item	The directory \emph{source/database/modelDatabase} contains the
  following files:
\begin{itemize}
\item \emph{dbmodel.mwb}: database model for opening in \emph{Mysql-Workbench}. 
\item \emph{dbmodel.mwb.bak}: backup of dbmodel.mwb.
\item \emph{modelDataBase}: database model in SQL language.
\item \emph{schema.png}: database schema picture.
\item \emph{setup.sh}: bash script to install and create the database in a \bonfire node.
\item \emph{Autorun.sh (Not used)}: bash script in order to start the mysql demon with a customized configuration.
\item \emph{my.cnf (Not used)}: MySQL service  customized configuration.
\end{itemize}
\item The directory \emph{source/database/scenaries} contains the following files:
\begin{itemize}
\item \emph{All\_Scenarios.csv}: contains the \ac{AOI} of all scenarios.
\item \emph{Scenario\_1\_Emergencies\_Lorca\_Earthquake.csv}: contains the visibility zones and the satellites that enters in them for the scenario 1.
\item \emph{Scenario\_2\_Infrastructure\_monitoring.csv}: contains the visibility zones and the satellites that enters in them for the scenario 2.
\item \emph{Scenario\_3\_South\_West\_England.csv}: contains the visibility zones and the satellites that enters in them for the scenario 3.
\item \emph{Scenario\_4\_Precision\_Agriculture\_Argentina.csv}: contains the visibility zones and the satellites that enters in them for the scenario 4.
\item \emph{Scenario\_5\_Basemap\_Worldwide.csv}: contains the visibility zones and the satellites that enters in them for the scenario 5.
\item \emph{setDatabase.py}: script that initializes the database created by
setup.sh.
\end{itemize}
\end{itemize}
\section{PlanetLab experiment files}
The resources corresponding \pl experiment are located in \emph{source/ple/} directory.
This directory contains quite files, and these are the following:
\begin{itemize}
\item \emph{iperfClients.py}: It contains the PlanetLab experiment script that obtains the client nodes bandwidth.
\item \emph{iperfClientsUDP.py}: It contains the PlanetLab experiment script that obtains the client nodes loss-rate.
\item \emph{pingClients.py}: It contains the PlanetLab experiment script that obtains the client nodes round trip time.
\item \emph{pPLE.py}: It contains the PlanetLab experiment script that obtains the ground station nodes bandwidth.
\item \emph{pPLEUDP.py}: It contains the PlanetLab experiment script that obtains the ground station nodes loss-rate.
\item \emph{pingPLE.py}: It contains the PlanetLab experiment script that obtains the ground station nodes round trip time.
\item \emph{bandwidth\_plotting\_nodes.py}: Script that plots the bandwidth nodes ordered by distance.
\item \emph{bandwidth\_plotting.py}: Script that plots the ground stations nodes bandwidth.
\item \emph{bandwith\_plotting\_node.py}: Script that plots the nodes bandwidth.
\item \emph{bandwidth\_plotting\_per\_ground.py}: Script that plots the ground stations nodes bandwidth into a histogram.
\item \emph{Loss-rate\_plotting\_nodes.py}: Script that plots the loss-rate per node.
\item \emph{Loss-rate\_plotting\_nodes\_customers.py}: Script that plots the loss-rate per client node.
\item \emph{pPLE.py}: 
\item \emph{pPLEUDP.py}:
\item \emph{*.out}: contains results of executions
\item \emph{Results* directories}: contains some execution results.
\end{itemize}