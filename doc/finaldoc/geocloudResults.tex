\section{GEO-Cloud Experiment Results}
\label{sec:geocloud-results}

The first results of the Cloud architecture were obtained from the execution of
the both first and second scenarios. The GeoCloud results were performed to
demonstrate that the architecture works as it should. 

The time of processing with the processors working in cloud with images located
into their local storage are shown in the Table~\ref{table:p-times}. The table
represents the times of each processor takes for processing an input
product. The first row of the table shows the real time of the processor wastes
in processing operations. The system time represents the time that the system
wastes performing the system calls that the processors do. The elapsed time is
the total time which the processing takes. The times of every I/O operations are
included in it.   

\begin{table}[!h]
  \centering
  {\small
  


\begin{tabular}{p{.13\textwidth}p{.13\textwidth}p{.13\textwidth}p{.13\textwidth}p{.13\textwidth}p{.13\textwidth}}
  \tabheadformat
  \tabhead{Processor}   &
  \tabhead{User Time \newline (s)}&
  \tabhead{System Time \newline (s)}&
  \tabhead{Elapsed Time \newline (HH:MM:SS)}&
  \tabhead{RAM}&
  \tabhead{CPU}\\

\hline
\textit{L0}         & 84.13 & 4.51 & 1:42.33&86\%&3.7\%\\
\hline
\textit{L0R}         & 100.34 & 17.34 & 1:50.72&106\%&23.26\%\\
\hline
\textit{L1A}         & 116.73& 17.24 & 1:48.94&122\%&12.2\%\\
\hline
\textit{L1BR}         &  1065.62& 568.48 & 17:42.64&153\%&26.4\%\\
\hline
\textit{L1BG}         & 1112.4 & 164.92 & 17:10.62&123\%&39\%\\
\hline
\end{tabular}


% Local variables:
%   coding: utf-8
%   ispell-local-dictionary: "castellano8"
%   TeX-master: "main.tex"
% End:

  }
  \caption{Processing times of each product processor}
  \label{table:p-times}
\end{table}

The L1CT and L1CR are not shown because these processors have not been installed
by platforms troubles.

Consecuently, these results are contrasted with the times obtained of executing
the processors in the shared storage. These results are depicted in Table~\ref{table:p-times-shared}
and there are much difference in terms of time.

\begin{table}[!h]
  \centering
  {\small
  


\begin{tabular}{p{.13\textwidth}p{.13\textwidth}p{.13\textwidth}p{.13\textwidth}p{.13\textwidth}p{.13\textwidth}}
  \tabheadformat
  \tabhead{Processor}   &
  \tabhead{User Time \newline (s)}&
  \tabhead{System Time \newline (s)}&
  \tabhead{Elapsed Time \newline (HH:MM:SS)}&
  \tabhead{RAM}&
  \tabhead{CPU}\\

\hline
\textit{L0}         & 88.43 & 3.67 & 4:15.06&36\%&3.7\%\\
\hline
\textit{L0R}         & 80.70 & 27.94 & 36:02.88&5\%&23.26\%\\
\hline
\textit{L1A}         & 104.75& 38.91 & 38:32.90&6\%&12.2\%\\
\hline
\textit{L1BR}         &  788.96& 14856.87 & 4:32:42&95\%&26.4\%\\
\hline
\textit{L1BG}         & 1030.16 & 1763.40 & 1:45:27&44\%&39\%\\
\hline
\end{tabular}


% Local variables:
%   coding: utf-8
%   ispell-local-dictionary: "castellano8"
%   TeX-master: "main.tex"
% End:

  }
  \caption{Processing times of each product processor in the shared storage}
  \label{table:p-times-shared}
\end{table}


The contrast is clear. The difference consists of the \ac{NFS} used is located
in other \emph{Fed4FIRE} testbed. It is \emph{IBBT} which is located in
Ghent(Belgium), so all the I/O operations such as reads and writes are performed
using the Internet network. The delays on the communications, the congestion of
the network and the packets distribution algorithms of the network generate that
huge difference between the times locally performed or the times obtained using
the shared storage.

To corroborate that hypotesis, a performance test in the shared storage was carried out. This test
consists of executing the following command:
\begin{itemize}
\item[>] dd if=/dev/zero of=/mnt/shared/test-performance bs=1M count=2560
  conv=fdatasync
\end{itemize}
It measures the write rate to the shared storage. The result obtained was the
following:
\begin{itemize}
\item 2684354560 bytes (2.7 GB) copied, 1174.6 s, 2.3 MB/s
\end{itemize}

Then, the command was repeated for executing in the local storage as follows:
\begin{itemize}
\item[>] dd if=/dev/zero of=/mnt/local/test-performance bs=1M count=2560
  conv=fdatasync
\end{itemize}

It measures the write rate to the shared storage. The result obtained was the
following:
\begin{itemize}
\item 2684354560 bytes (2.7 GB) copied, 50.975 s, 52.7 MB/s
\end{itemize}

Thus it is concluded that the bottleneck obtained in the products processors
implemented in the shared storage is due to the implementation of the \ac{NFS}
storage which the
\emph{Fed4FIRE} testbed has implemented.

Furthermore, the cataloguing time of any image is near to cero in the local
implementation of the storage and it is nil in the implementation of the shared
storage. In the first one exists a data transfer  between the \emph{Processing
  Chain} and the \emph{Archive and Catalogue} modules. Althought
the data transfer rate between \bonfire machines is 1 Gbps, so the transferring
time is short. However in the second one, these transfers are avoided so the
storing and archiving is inmediately done.

Finally, a video of a real time system execution is included in the CD-ROM.

