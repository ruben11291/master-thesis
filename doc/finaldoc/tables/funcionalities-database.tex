

\centering
{\small
\begin{longtable}{p{.2\textwidth}p{.2\textwidth}p{.2\textwidth}p{.2\textwidth}}
  \tabheadformat
  \tabhead{Table}   & \tabhead{Function}& \tabhead{Columns} &
  \tabhead{Relationship}\\\hline
\endhead

\textit{Scenarios table}    &  It contains the name of the scenario, the start
time $T_0$ and the end time $T_f$. Its primary key\footnote{The \emph{primary
    key} identifies an object; for example a scenario, a ground station or a
  satellite} (represented with a key symbol in
Figure~\ref{fig:sss-database-architecture}) is the name column.   &
name \newline timeIni \newline timeEnd&None\\
\hline


\textit{GroundStations table} & It contains the ID of the Ground Station and its
name. Its \emph{primary key} is the idGroundStation. & idSatellite  \newline scenario  \newline idGroundStation  \newline timeInStation  \newline timeOutStation
 \newline interestZoneIni  \newline interestZoneEnd &None  \\\hline

\textit{Satellites table} &  It contains the ID of the satellites
\emph{idSatellite}, the scenario and the ground station id,
\emph{idGroundStation}, in which the events occur, the time in which the
satellite enters into the visibility cone $[T_{GS}]_{0ij}$,
\emph{timeInStation}, the time when it leaves the visibility cone
$[T_{GS}]_{fij}$, \emph{timeOutStation}, the time when the satellite enters into
the \emph{AIO} $[T_{AOI}]_{0i}$, \emph{interestZoneIni}, and the time when the
satellite leaves the \emph{AOI} $[T_{AOI}]_{fi}$,\emph{interest}~\emph{ZoneEnd}. All of
them are primary keys because a satellite can download images of several
\emph{AOIs} in a single ground station.
  & idGroundStation \newline name \newline ip \newline port& This table relates
  the \emph{Scenarios} table and the \emph{GroundStations} table. This is
  because the \emph{scenario} and \emph{idGroundStation} columns are foreign
  keys\footnote{The \emph{foreign key} binds two tables, linked across a
    field. For example in this case, the column from the \emph{Satellites} table
    which is named \emph{Scenario} refers to an object that exists in the
    \emph{Scenarios} table.}. The characteristics of these columns are \emph{On
    delete cascade} and \emph{On update cascade}. This means that when an object
  contained in a table (\emph{GroundStation} table or \emph{Scenarios} table) is
  deleted, it is also automatically deleted the object that refers to it, and if
  it is updated the object is also automatically updated. \\\hline
\caption{Funcionalities of the database tables for the simulator} \label{table:sss-funcionalities-database}
\end{longtable}
}

% Local variables:
%   coding: utf-8
%   ispell-local-dictionary: "castellano8"
%   TeX-master: "main.tex"
% End:
