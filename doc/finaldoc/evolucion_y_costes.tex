\chapter{Evolution and Costs}
\label{chap:evolution}
\drop{T}{he} different phases performed in the development of this project are described,
specifying the milestones of each one and their temporal cost and complexity.


\section{Project Evolution}
\label{section:evolution}

Before the start of the development phase (March 5th), a training stage with
the different testbeds used for the development, the specifically technologies of
\emph{Elecnor Deimos} company and some Spatial acknowledge were studied during a month. Then the
requirement analysis stage such as technological and functional terms were
studied. Several papers describing the component-oriented architecture were
performed as a result. Once the requirements were specified, the iterative
development started. Below these phases are fully described.


\subsection{Preliminary requirements analysis}

The initial requirements analyses which the entire system should satisfy were
obtained after a week from the end of training stage.
At first, some basics requirements were fixed. These are the following:
\begin{itemize}
\item It must base in open source and open standards.
\item The language used for the implementation must be multiplatform, dynamic
  and multipurpose.
\item The tools used for development should be the \emph{Fed4FIRE} tools.
\item The platforms were the software must be checked were the \emph{Fed4FIRE}
  platforms.
\item The implemented system must be as realistic as possible. This means to
  create a \emph{Ground Segment} such as it is owned by \emph{Elecnor Deimos}
  for checking and measuring the performance which the same \acl{EO} system may
  expand in a cloud platform.
\end{itemize}

Then a document containing the requirements of the whole system and each
component of it were accomplished. In this document the components of the
\emph{Space System Simulator},
the cloud architecture and the experiment in \pl were stated. The requirements
of each component of
the \sss such as the \satss and \gsss were obtained. The need the inclusion of
the database into system were discovered because there were the only way to obtain the
\ac{IP} of the \emph{Ground Stations}. The cloud architecture components were
individually analysed, wherewith the communications between them and its
interactions. The requirements of the different machines where these serves
were taken place, the network requirements and topologies were studied.  
The requirements of the \pl experiment basically consist of which impairments
are involved in network communications. They were studied and investigated for
knowing the influence over the transmission speed, loss rate data and latency.

Then the design of the scenarios and their executions were defined in order to
check all functionalities and to evaluate the performance which the experiment must offer.
Moreover the \ac{GUI} was not taken into account but at the moment the
experiments started, it was decided to design and to implement the graphical interface.


\subsection{General design}

The design of GeoCloud project was done inside a modular and extensible
frame. It is composed by many components which each one of them is
self-contained, in different platforms and interfaces. Transparency, openness
and scalability are the adjectives which governed the development. Thus if were
necessary to introduce or to modify any functionality, the effort to do this will
be minimal.
% In principle the
%\emph{Fed4FIRE} federation facilitates their communications and operations but,
%at the moment of the implementation of this project, some funcionalities did not
%properly work. 

Furthermore, it was interesting to automatize the execution of the
experiments. For that the \ac{GUI} was developed in order to any user without acknowledges about the implementation, could to use and understand it.

Some software engineering patterns were applied such as \emph{Proxy},
\emph{Facade}, \emph{Controller} and \emph{Singleton}. These patterns allow to understand and easily to
maintain the source code. 

\subsection{Iterations of the development}

In this section the iterations done for the development of GeoCloud are
enumerated and explained describing the achievements, the complexity of some
parts and the most relevant decisions considered.

\subsubsection{Iteration 1}

Firstly the folder structure in which the source and the documentation are
located, were created. The scale of the project is quite considerable so it was
strictly necessary to establish a main directory per each testbed. Furthermore,
a pair of accounts were created; the first one, in the \bonfire official page
and other one, in the \emph{iMinds Emulab} official page. In both profiles, the
\ac{RSA} key generated was uploaded.
For the designing of the entire cloud architecture, a top-down model was
selected. Thus, the Ground Segment on premises of \emph{Elecnor Deimos} was
studied and the main components for obtaining geolocated images of the Earth
surface were identified. These cloud architecture components were also designed
studying its relationships and the data flows between them. The design of the
interfaces and relationships between all cloud components were analysed. 
The basic architecture of the Orchestrator component were
designed, drafted and some \ac{UML} diagrams were pictured. 

Regarding the \sss, the data files of the defined scenarios such as satellite accesses,
time of each satellite is inside a footprint and the numbers of acquired images per
satellite were obtained. Then, a data analysis of them was carried out in order
to design and build the data base and the script for populating it.


The main objective for this iteration can be summarized such as the preparation
of the development environment and the design of the cloud system and the
obtaining data of the satellite constellation in order to build a customized simulator.

\subsubsection{Iteration 2}

The implementation of the cloud architecture components was started. First, the
Orchestrator component was developed in order to locally run. The class
Processing Chain controller was designed using a Singleton pattern. Then, the
Archive and Catalogue module was also locally implemented. \emph{Geo-Server} and
\emph{Geo-Network} were studied in order to obtain which one was more convenient
for deploying into the component. \emph{Apache Tomcat} and \emph{Geo-Server}
were installed and the \ac{CSW} plugin of \emph{Geo-Server} was also deployed
and tested.
The product processors of \emph{Elecnor Deimos} were studied and installed in
the development computer. The Processing Chain component was implemented and
tested executing
via command line the processors respectively.

For implementing the \sss, some Python libraries for scheduling and planning
actions were studied. Then, the development of the \emph{Satellite System
  Simulator} was carried out obtaining the satellite data from database and
scheduling functions depending on the behaviour of the satellite in each
scenario.
The development of the \gsss was done conceiving the \emph{Ground Stations
  Simulators} as simple servers processing the received packets.   
At first, the \emph{Satellite System Simulator}  sent a raw data in a constant
bit rate. However, the required bit-rate was $160~Mbps$, thus the network over internet
can not sure it transmission rate. The solution was to implement it sending tiny
packets in order to the network was able to transmit it in the time required.    
Using this approach, the \emph{Ground Stations
  Simulators} taken into account the number of packets received. The number of
the received images by the ground stations depended on
the number of these packets. Several test cases were done in order to validate
this approach.

At the end of the iteration 2, the \sss was finished and tested and the cloud
architecture was locally implemented and their components individually tested.
 
\subsubsection{Iteration 3}

In this iteration, the main objective consists of designing and implementing of the
\pl experiment. For that, the federation tools provided by the \emph{Fed4FIRE}
project were investigated. As a result, \emph{Nepi} was selected for doing the
profiling tool in \pl. 
For the development of this experiment, the following stages were performed:
\begin{itemize}
\item Searching and selecting nodes around the world for simulating the
  end-users and ground stations.
\item Testing the connectivity of all selected nodes and creating some scripts
  for testing them. This process was complicated because the most of nodes
  appear in the \pl webpage as available and working but executing the script,
  some of them were unreachable. Thus, this trouble delayed the experiment
  because it was necessary to check node to node for only selecting the available.
\item Deliberating about the impairments for measuring (loss-rate, effective
  bandwidth and delay) and with which tools these measures can be obtained. 
\item Finally, the implementation and execution of the developed scripts were accomplished.
\end{itemize}

Once the results were obtained, some scripts using \emph{Python-matplotlib} were
developed. These scripts read the collected data and represented them using bar
plots or linear plots. 

\subsubsection{Iteration 4}

The \sss was not implemented yet in \vw testbed. So using \emph{JFed}, a
deployment of \vw nodes was accomplished. There were lots of problems with that
because the nodes connectivity was correct for \ac{IP}v6 but not for
\ac{IP}v4. Once the nodes deployment was successful and they can be reached via
\ac{SSH}, the \sss was installed into them.  

Then the architecture in \bonfire was also implemented and integrated. The
machines which harboured the cloud components were prepared. The \ac{FTP} server
into the Orchestrator, the \emph{GeoServer} software and \emph{Apache} installed
into the \emph{ArchiveAndCatalogue} and the product processors were laid into
the \emph{Chain Processing} machine.

Once the cloud architecture was prepared, the software was uploaded. Then, the
scenario 1 was selected for testing the system and it was executed. There were
to do some changes in the code in order to correct some bugs. At first
these executions were manually made, however there were some task and
interactions with machines and it was necessary to develop an user interface to
automatize all processes.

Finally, the \ac{GUI} was performed and the execution of these experiments were
perfectly automatized. Furthermore, the \emph{European Commission} invite us to
present this project and for doing a demonstration who the system worked and how
the results of these executions can be seen at real time by the end-user.

At the end of this iteration, the \sss was fully implemented and deployed. The
cloud architecture was built and all its components interconnected and executing
scenarios. These scenarios provided the integration test of the entire cloud
system implementation and the demonstration of the functionality in front of the
\emph{European Commission}.


\subsubsection{Iteration 5}

In this iteration the main issues were the experiment executions and the
collection of some results repeatedly. 

The scenario 1 and 2 were carried through for obtaining results. This results
are in Section~\ref{sec:geocloud-results}. Next the implementation of a shared
storage in order to avoid the sends between the cloud components was done. This
shared storage is based in \ac{NFS}. This protocol enables sharing files between
some compute resources and it maintains the consistency of these files.
The experiment also was executed with this latest enhancement obtaining results
for contrasting with the results obtained without shared storage. This
comparative can be also seen in the results section.


\subsubsection{Iteration 6} 

The last step of this project was to implement all components of Cloud system using
a distributed middleware as ZeroC ICE. First, the slice file which contains the
definition of the distributed interfaces were carried out. Then, the
implementation of each individual component were performed implementing the
operations which were defined in the slice file. During the development of this
architecture, lots of manually test were performed. Once the implementation
finished and the test cases were successfuly executed,  IceGrid were used for
the automatically deployment between all the distributed nodes. The nodes
implementation and deployment using IceGrid was locally tested. 
Next, the arquitecture had to be installed and executed on cloud. The port
policies restrictions avoid this architecture over ICE middleware was tested and
checked in the cloud platform.
 
\section{Resources and costs}

This section lists and describes the various resources used, both temporal 
and economically (for the latter an estimate was performed). Furthermore, the
statistics of the employed repository for control of versions are presented.

\subsection{Economical cost}

The training time of this project comprises from 3rd of February to 10th of
March (24 days). The approximated daily dedicated time is 8 hours and the considered price for that period of time was \EUR{5}/hour. The approximated dedicated time for implementation was 8 hours daily
during 105 days which add in a total of 840 hours. It was considered an average
price of \EUR{15}/hour a programmer (reference prices acquired from
\url{https://freelance.infojobs.net/freelancers})

The Table~\ref{table:economical-breakdown} shows the breakdown of resources used during the development.


\begin{table}[hp]
  \centering
  {\small
  


\begin{tabular}{p{.2\textwidth}p{.2\textwidth}p{.2\textwidth}}
  \tabheadformat
  \tabhead{Resource}   &
  \tabhead{Amount}&
  \tabhead{Cost}   \\
\hline
\textit{Trainning salary} & 1 & \EUR{960} \\\hline
\textit{Programmer salary}  & 1   & \EUR{12600} \\
\hline
\textit{Intel Core i5 3450 3.1 GHz 8GB RAM}     & 1 & \EUR{420} \\
\hline
\textit{European Commision Review}    & 1 & \EUR{630} \\
\hline
\textit{FOSS4G Conference}   &1      & \EUR{720} \\
\hline
\textit{TOTAL} & & \EUR{15330}\\ 
\hline
\end{tabular}


% Local variables:
%   coding: utf-8
%   ispell-local-dictionary: "castellano8"
%   TeX-master: "main.tex"
% End:

  }
  \caption{Economical breakdown for the GeoCloud development}
  \label{table:economical-breakdown}
\end{table}

\subsection{Repository statistics}

The version control tool used in the development of this project was \emph{Git}
supported by the \emph{Bitbucket} repository service.

\emph{Cloc} tool was used for counting the source lines of code of all languages in which
GeoCloud was developed. This tool counts all lines of the code differentiating
between code, blank lines and comment lines properly. The Table~\ref{table:git-statistics} shows the
execution results of \emph{Cloc} in the source folder.

\begin{table}[hp]
  \centering
  {\small
  


\begin{tabular}{p{.2\textwidth}p{.2\textwidth}p{.2\textwidth}p{.2\textwidth}p{.2\textwidth}}
  \tabheadformat
  \tabhead{Languaje}   &
  \tabhead{Files}&
  \tabhead{Blank \newline spaces}   &
  \tabhead{Comments}   &
  \tabhead{Source lines \newline of code}   \\

\hline
\textit{Python}  & & &  & System library \\
\hline
\textit{XML}    & & & & \\
\hline
\textit{Bourne Shell} & & & & \\
\hline
\textit{Make}         & & & & \\
\hline
\end{tabular}


% Local variables:
%   coding: utf-8
%   ispell-local-dictionary: "castellano8"
%   TeX-master: "main.tex"
% End:

  }
  \caption{Number of source lines of code of project}
  \label{table:git-statistics}
\end{table}
