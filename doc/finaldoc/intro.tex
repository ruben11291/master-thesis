\chapter{Introducción}

\drop{E}{arth} Observation (EO) commercial data sales have increased a 550\% in
the last decade \cite{sousa}[1]. This area is considered a key element in the
space industry and an opportunity market for the next years. However, processing
and distribution of large-sized data obtained from satellite recordings still presents critical challenges. 
EO industries implement on-site conventional infrastructures to acquire, store, process and distribute the geo-information generated. However these solutions have the risks of over/under size the infrastructure, they are not flexible to cover sudden changes in the demand of services and the access to the information presents large latencies. 

In addition, new sectors and user typologies are applying for new EO services
and there is an incresing demand of this services. These users
need more flexible, easy and instant access to EO products and services through
the Web. This demand has traditionally been driven through Space Data
Infrastructures and heavy standards (ISO TC/211 and OGC) which are focused on
interoperability rather than the real demand from the end-users. 

To provide these services, in Geo-Cloud project a cloud architecture is
implemented. This way can solve the previously limitations because the cloud features are elasticity, transparency, scalability and on-demand use characteristics [2]. 

The Geo-Cloud experiment has been designed and implemented in the European federated framework Fed4FIRE [3], [4] to test cloud computing for processing, storing and distributing geo-information on demand, through web services, by using open source geo-software.
	
The GEO-Cloud experiment utilizes the free platforms PlanetLab (composed by
PlanetLab Europe and PlanetLab Central) [5], [6], Virtual Wall [7], [8] and
BonFIRE [9], [10] future internet resources to go beyond conventional services
and infrastructures in the EO sector and implement and test in cloud a complete
EO system. In the PlanetLab facility, the Ground Stations and the cloud are
modeled in order to simulate the communications between them and to obtain the
real network impairments. Then, these impairments are used to update the
required impairments by BonFIRE and Virtual Wall.
In  

\section{Earth Observation}

La observación de la Tierra ha supuesto todo un reto desde que el hombre tiene
conocimiento de la existencia de ella. Este interés proviene de la intención de
conocer todo a nuestro alrededor y poder usarlo en beneficio. 
La existencia de los satélites de observación
de la tierra facilitan esta tarea debido a que en una simple imágen captada por
ellos, se pueden vislumbrar de un vistazo, kilómetros y kilómetros de la
superficie terrestre. Su existencia proporcionan una herramienta muy útil para
organismos públicos, privados o incluso para realizar nuestra vida
cotidiana. Entre los campos en los que los satélites tienen mucha importancia
por ejemplo para la obtención de mapas del territorio, conocer el estado de las
carreteras, servicios de emergencia, monitorización ambiental, agricultura de
precision, control de infraestructuras, étc. 

\subsection{Socio-Economical Impact Analysis}

The Geo-Cloud project is used as a framework to offer services from EO
users. The benchmark developed in the experiment allows to establish the
frontiers of viable and not viable cloud solutions in EO depending on  

In the last decade, a large cantity of companies have started 
\section{PlanetLab}

This testbed consists of several PCs interconnected around the world forming a
network. This nodes are used to emulate the real behaviour of the links between
En la portada ---y otras páginas de presentación--- el nombre o título del
proyecto debe aparecer sin comillas, cursiva u otros formatos. Si se cita el
título de otra obra, o el nombre de un capítulo sí debe aparecer entre
comillas. Por cierto, las comillas que deben usarse en castellano son las
«latinas», dejando las ``inglesas'' para los raros casos en los que aparezca una
cita en el cuerpo otra~\cite{sousa}.


\section{Estructura del documento}

Pueden incluirse aquí una sección con algunos consejos para la lectura del
documento dependiendo de la motivación o conocimientos del lector.  También
puede ser útil incluir una lista con el nombre y finalidad de cada uno de los
capítulos restantes.


\begin{definitionlist}
\item[Capítulo \ref{chap:antecedentes}: \nameref{chap:antecedentes}] Explica herramientas
  y aspectos básicos de edición con \LaTeX.
\item[Capítulo \ref{chap:objetivos}: \nameref{chap:objetivos}] Finalidad y justificación
  (con todo detalle) del presente documento.
\end{definitionlist}


\section{Más texto para que ocupe varias páginas}

\input{lorem-ipsum.txt}
